% arara: xelatex
% arara: xelatex
% arara: xelatex


% options:
% thesis=B bachelor's thesis
% thesis=M master's thesis
% czech thesis in Czech language
% english thesis in English language
% hidelinks remove colour boxes around hyperlinks

\documentclass[thesis=B,english]{FITthesis}[2019/12/23]

%\usepackage[utf8]{inputenc} % LaTeX source encoded as UTF-8
% \usepackage[latin2]{inputenc} % LaTeX source encoded as ISO-8859-2
\usepackage[cp1250]{inputenc} % LaTeX source encoded as Windows-1250

% \usepackage{subfig} %subfigures
% \usepackage{amsmath} %advanced maths
% \usepackage{amssymb} %additional math symbols

\usepackage{dirtree} %directory tree visualisation

% % list of acronyms
% \usepackage[acronym,nonumberlist,toc,numberedsection=autolabel]{glossaries}
% \iflanguage{czech}{\renewcommand*{\acronymname}{Seznam pou{\v z}it{\' y}ch zkratek}}{}
% \makeglossaries

% % % % % % % % % % % % % % % % % % % % % % % % % % % % % % 
% EDIT THIS
% % % % % % % % % % % % % % % % % % % % % % % % % % % % % % 

\department{Department of Applied Mathematics}
\title{Detecting abnormalities in X-Ray images using Neural Networks}
\authorGN{Uladzislau} %author's given name/names
\authorFN{Yorsh} %author's surname
\author{Uladzislau Yorsh} %author's name without academic degrees
\authorWithDegrees{Uladzislau Yorsh} %author's name with academic degrees
\supervisor{Ing. Jakub {\v Z}itn{\' y}}
% \acknowledgements{THANKS (remove entirely in case you do not with to thank anyone)}
\abstractEN{Rheumatoid arthritis (RA) is a systemic inflammatory autoimmune disease affecting 0.5 to 1 \% of adult people in the developed world. This disease attacks the synovium, results in swollen and painful joints and, in severe cases, leads to disability. The current approach in diagnosing and quantifying damage is a manual radiographic image inspection by a radiologist, which is generally expensive, time-consuming and subjective. Automated assessment systems are a way to overcome this problems and introduce more objectivity into radiology reports, coupled with a faster damage quantifying.
	
The possibility of such a system development will be shown on the example of a participation in RA2 DREAM Challenge. As a part of the competition, the system utilizing several state-of-art convolutional neural net architectures will be developed and scored on the University of Alabama at Birmingham Cheaha supercomputing system.}
\abstractCS{Revmatoidn{\' i} artritida (RA) je syst{\' e}mov{\' e} z{\' a}n{\v e}tliv{\' e} autoimunitn{\' i} onemocn{\v e}n{\' i}, postihuj{\' i}c{\' i} 0,5 a{\v z} 1 \% dosp{\v e}l{\' y}ch lid{\' i} v rozvinut{\' e}m sv{\v e}t{\v e}. Dan{\' a} nemoc vyvol{\' a}v{\' a} chronick{\' y} z{\' a}n{\v e}t synovi{\' a}ln{\' i} tk{\' a}n{\v e}, co{\v z} vede k nevr{\' a}tn{\' e}mu po{\v s}kozen{\' i} chrup{\' a}vek a kost{\' i} kloub{\r u}. Sou{\v c}asn{\' y} p{\v r}{\' i}stup k diagnostice a hodnocen{\' i} po{\v s}kozen{\' i} je vizu{\' a}ln{\' i} inspekce rentgenov{\' y}ch sn{\' i}mk{\r u} radiologem, kter{\' a} je obecn{\v e} drah{\' a}, {\v c}asov{\v e} n{\' a}ro{\v c}n{\' a} a subjektivn{\' i}. Aumtomatick{\' e} hodnot{\' i}c{\' i} syst{\' e}my jsou jedn{\' i}m ze sp{\r u}sob{\r u} tyto probl{\' e}my p{\v r}ekonat a zav{\' e}st do radiologick{\' y}ch zpr{\' a}v v{\' i}ce objektivity spolu s rychlej{\v s}{\' i}m vy{\v c}{\' i}slen{\' i}m {\v s}kody.

Mo{\v z}nost vytvo{\v r}en{\' i} takov{\' e}ho syst{\' e}mu bude uk{\' a}z{\' a}na na p{\v r}{\' i}kladu {\' u}{\v c}asti v RA2 DREAM Challenge. V r{\' a}mci sout{\v e}{\v z}e na z{\' a}klad{\v e} n{\v e}kolika modern{\' i}ch architektur konvolu{\v c}n{\' i}ch neuronov{\' y}ch s{\' i}t{\' i} bude vyvinut syst{\' e}m, kter{\' y} pozd{\v e}ji bude ohodnocen na vypo{\v c}etn{\' i}m clusteru Birmingham Cheaha na University of Alabama.}
\placeForDeclarationOfAuthenticity{Prague}
\keywordsCS{strojov{\' e} u{\v c}en{\' i}, neuronov{\' e} s{\' i}t{\v e}, automatizovan{\' e} hodnocen{\' i}, medic{\' i}nsk{\' y} software}
\keywordsEN{machine learning, neural networks, automated assessment, medical software}
\declarationOfAuthenticityOption{4} %select as appropriate, according to the desired license (integer 1-6)
% \website{http://site.example/thesis} %optional thesis URL


\begin{document}

% \newacronym{CVUT}{{\v C}VUT}{{\v C}esk{\' e} vysok{\' e} u{\v c}en{\' i} technick{\' e} v Praze}
% \newacronym{FIT}{FIT}{Fakulta informa{\v c}n{\' i}ch technologi{\' i}}

\setsecnumdepth{part}
\chapter{Introduction}

Motivation and the work objectives

\setsecnumdepth{all}
\chapter{Background}

Description of the techniques used

\section{Computer Vision}

CV and medicine
	
\section{Machine Learning}

What the ML is, supervised/unsupervised learning, training data

\section{Convolutional Neural Nets}

Pros and cons, ConvNets, their distinction from ANNs

\section{Overfitting and Regularization}

What the overfitting and regularization are, smoother function example, how to combat the problem

\chapter{State-of-the-art}

Review of the modern CNNs, ideally with examples of appliance in medicine

\chapter{Analysis and design}

Data analysis and model designing

\chapter{Realisation}

Frameworks and implementations used

\chapter{Experiments}

Numeric experiment results with graphs and tables

\setsecnumdepth{part}
\chapter{Conclusion}

Pros and cons of the approach, what could be done better

\bibliographystyle{iso690}
\bibliography{mybibliographyfile}

\setsecnumdepth{all}
\appendix

\chapter{Acronyms}
% \printglossaries
\begin{description}
	\item[GUI] Graphical user interface
	\item[XML] Extensible markup language
\end{description}


\chapter{Contents of enclosed CD}

%change appropriately

\begin{figure}
	\dirtree{%
		.1 readme.txt\DTcomment{the file with CD contents description}.
		.1 exe\DTcomment{the directory with executables}.
		.1 src\DTcomment{the directory of source codes}.
		.2 wbdcm\DTcomment{implementation sources}.
		.2 thesis\DTcomment{the directory of \LaTeX{} source codes of the thesis}.
		.1 text\DTcomment{the thesis text directory}.
		.2 thesis.pdf\DTcomment{the thesis text in PDF format}.
		.2 thesis.ps\DTcomment{the thesis text in PS format}.
	}
\end{figure}

\end{document}
